\documentclass[10pt,twoside,letterpaper]{article}

\usepackage{times}
\usepackage{amsmath}
\usepackage{amssymb}
\usepackage{graphicx}
%\usepackage{fullpage}
\usepackage{booktabs}
\usepackage{paralist}
\usepackage{anysize}
%\marginsize{29mm}{29mm}{26mm}{26mm}
\marginsize{25mm}{25mm}{18mm}{18mm}

\newcommand{\selffont}[1]{{\textit{#1}}}
\newcommand{\venuefont}[1]{{\textit{#1}}}
\newcommand{\negitemspace}{\vspace{1mm}}
\newcommand{\tabrowsep}{\vspace{1mm}}
\newcommand{\myself}{\selffont{L. Li}}
\newcommand{\scheduled}{\textit{[scheduled]}}

\usepackage{fancyhdr}
\pagestyle{fancy}
% with this we ensure that the chapter and section
% headings are in lowercase.
\fancyhf{} % delete current setting for header and footer
\fancyhead[LE,RO]{Lihong Li} %
\fancyhead[LO]{Google} %
\fancyhead[RE]{\today} %
\renewcommand{\headrulewidth}{0.5pt}
\renewcommand{\footrulewidth}{0pt}
\addtolength{\headheight}{0.5pt} % make space for the rule
\renewcommand{\footrulewidth}{0.4pt}
\fancyfoot[LE,RO]{\thepage}
\fancypagestyle{plain}{%
\fancyhead{} % get rid of headers on plain pages
\renewcommand{\headrulewidth}{0pt} % and the line
\fancyfoot{}
}

\begin{document}

\title{\textmd{Lihong Li}}
%\author{lihong@cs.rutgers.edu \\
%  http://www.cs.rutgers.edu/\~{}lihong \\
%  Rutgers University, Piscataway, NJ 08854}
\date{}
\maketitle

%\vspace{-45mm}
%\noindent \makebox[\columnwidth][l]{\includegraphics[width=25mm]{RU_INF_SEAL_CMYK}}
%\vspace{15mm}

\vspace{-15mm} \noindent \makebox[\columnwidth][c]{
  \parbox[l]{0.35\columnwidth}{
    Google Inc. \\
    747 Sixth Street South \\
    Kirkland, WA, USA 98033
  }
  \parbox[r]{0.6\columnwidth}{
    \begin{flushright}
    lihong@google.com \\
    lihongli.cs@gmail.com \\
    https://lihongli.github.io
%    http://www.cs.rutgers.edu/$\sim$lihong
%    Last updated: \today
    \end{flushright}
  }
}

\vspace{-5mm}
\begin{center}
\begin{tabular}{p{160mm}}
\toprule \\
\end{tabular}
\end{center}
\vspace{-8mm}

%\subsection*{\textsc{\underline{Objective}}}
%
%\hspace{13mm} A summer internship position in applications or
%algorithmic analysis/development in the areas of reinforcement
%learning, machine learning, pattern recognition, planning, or
%related fields.

\subsection*{\textsc{\underline{Research Interests}}}


My core research interest is in \textbf{machine learning for interactive systems that maximizes a utility function by taking
actions}, which is in contrast to prediction-oriented machine learning like supervised learning. My area of focus is
\textbf{reinforcement learning}, including \textbf{contextual bandits}, and I am also interested in related areas such as large-scale learning, active learning, and planning. In the past, I have applied my work to recommendation,
Web search, advertising, conversation systems, and spam detection.

\subsection*{\textsc{\underline{Education}}}

\begin{center}
%\begin{tabular}{lll}
\begin{tabular}{p{30mm} p{15mm} p{115mm}}
01/2005 -- 05/2009 & Ph.D. & Computer Science, Rutgers University, USA \\
%                   &       & Title: A Unifying Framework for Computational Reinforcement Learning \\
%                   &       & Advisor: Michael L. Littman \tabrowsep \\
09/2002 -- 07/2004 & M.Sc. & Computing Science, University of Alberta, Canada \\
%                   &       & Co-advisors: Vadim Bulitko \& Russell Greiner \tabrowsep \\
09/1998 -- 07/2002 & B.Eng. & Computer Science and Technology, Tsinghua University, China %
\end{tabular}
\end{center}

%\subsection*{\textsc{\underline{Research Interests}}}
%
%My PhD research focuses on \textbf{mathematical analysis of
%sequential decision making algorithms}.  I am financially
%supported by funding from the \textbf{National Science Foundation
%(NSF)}, supervised by \textbf{Prof. Michael Littman} at the
%\textbf{Rutgers University}. My representative scientific
%contributions include:
%%
%\begin{itemize}\addtolength{\itemsep}{-0.1in}
%%
%\item{Proposed an algorithm for solving sequential decision making
%problems with continuous variables; it enjoys strong theoretical
%guarantees and is computationally more efficient than standard
%approaches (published in AAAI'2005).}
%%
%\item{Developed a general framework for state abstraction in
%Markov decision processes that unifies almost all previously
%proposed notion of state abstraction (published in
%AI\&Math'2006).}
%%
%\item{Proposed the state-of-the-art algorithm and proved it
%converges to near-optimal solutions in polynomially many steps
%(published in ICML'2006).}
%%
%\item{Proposed two provably efficient model-based algorithms that
%enjoy better running-time efficiency than previous algorithms
%(published in UAI'2006).}
%%
%\item{Analyzed and proved that a family of popular algorithms used
%for constructing features in sequential decision making is sound
%as they consistently lead to error reduction (published in
%ICML'2007).}
%%
%\item{Used linear algebra and matrix theory to compare two popular
%algorithms in sequential decision making, which leads to useful
%insights of their relative strengths (published in ICML'2008).}
%%
%\item{Proved a fundamental equivalence of two important classes of
%algorithms, which gives important insights into many existing
%feature-selection algorithms (published in ICML'2008).}
%%
%\item{Proposed a new mathematical framework called KWIK that
%covers and unifies a number of existing provably efficient
%algorithms (published in ICML'2008).}
%%
%\item{Developed and experimented with a highly efficient algorithm
%for solving large-scale machine-learning problem (to appear in
%NIPS'2008).}
%%
%\item{Proved that a classic algorithm, proposed $15$ years ago,
%can indeed converge to the optimal solution when it is run forever
%(under review, 2008).}
%%
%\end{itemize}

\subsection*{\textsc{\underline{Research \& Industry Experience}}}

\begin{tabular}{p{30mm} p{120mm}}
10/2017 -- present & Research Scientist at Google (Kirkland) \\
03/2017 -- 10/2017 & Principal Researcher at Microsoft Research (Redmond) \\
02/2016 -- 02/2017 & Senior Researcher at Microsoft Research  (Redmond) \\
06/2012 -- 02/2016 & Researcher at Microsoft Research (Redmond) \\
%                   & {\small\textmd{Research and publications in multi-armed bandits and reinforcement learning, with applications in Web search, spoken dialog systems, and other domains.  Helped maintain connection between Microsoft Research and academia by actively publishing papers, serving on conference committees, giving talks, and visiting labs/universities, etc.}} \\
09/2010 -- 06/2012 & Research Scientist at Yahoo! Research (Santa Clara) \\
06/2009 -- 08/2010 & Postdoctoral Scientist at Yahoo! Research (Santa Clara) \\
%                   & {\small\textmd{Research and publication of machine-learning theory and techniques, with applications to the Internet; focus was on problems with interactive nature, such as multi-armed bandits, with applications to online advertising, content recommendation, ranking, and spam detection.}} \tabrowsep \\
06/2008 -- 08/2008 & Research Intern at AT\&T Shannon Labs \\
%                   & {\small\textmd{Research on scaling up reinforcement learning and feature selection in spoken dialog systems.}}\tabrowsep \\
05/2007 -- 08/2007 & Research Intern at Yahoo! Research NYC \\
%                   & {\small\textmd{Research on large-scale sparse learning and exploration/exploitation tradeoff in sponsor search.  Helped develop Web-scale learning algorithms in the Vowpal Wabbit project.}}\tabrowsep \\
05/2006 -- 08/2006 & Engineering Intern at Google NYC \\
%                   & {\small\textmd{Design and implementation of software modules for an object identification task.}}\tabrowsep \\
01/2005 -- 05/2009 & Graduate Research Assistant at the Rutgers University \\
%                   & {\small\textmd{Research on models and algorithms for reinforcement learning, decision-theoretic planning, and machine learning.}}\tabrowsep \\
09/2002 -- 07/2004 & Research Assistant at the University of Alberta \\
%                   & {\small\textmd{Research on algorithms for reinforcement learning, as well as machine-learning techniques in an automatic object recognition system.}} %
\end{tabular}

\subsection*{\textsc{\underline{Selected Awards}}}

\begin{center}
\begin{tabular}{p{10mm} p{10mm} p{130mm}} % {c l l}
%1995 & China & 1st Class Prize, National Competition in Mathematics (Junior High) \tabrowsep \\
%1995 & China & 1st Class Prize, National Competition in Physics (Junior High) \tabrowsep \\
%1997 & China & 1st Class Prize, National Competition in Mathematics (Senior High) \tabrowsep \\
%1997 & China & 2nd Class Prize, National Competition in Physics (Senior High) \tabrowsep \\
%1997 & China & 2nd Class Prize, National Competition in Chemistry (Senior High) \tabrowsep \\
%1999 & China & Excellent Freshman Scholarship, Tsinghua University \tabrowsep \\
%2001 & China & Honor for Excellent Work in Departmental Student Union, Tsinghua University \\
%2003 & Canada & J. Gordin Kaplan Scholarship, University of Alberta \tabrowsep \\
%2003 & Canada & GSA Professional Development Grant, University of Alberta \tabrowsep \\
%& 2004 & Mexico & Student-Author Travel Scholarship, IJCAI'2005  \\
%2004 & Canada & {Teaching Assistant Award, University of Alberta} \tabrowsep \\
%& 2005 & USA & Student-Author Travel Scholarship, AAAI'2005 \\
%& 2006 & USA & Student-Author Travel Grant, National Science Foundation \\
%2006 & USA & Winner in \textsc{Pentathlon} and first place in \textsc{PuddleWorld}, First Annual Competition of Reinforcement Learning (w/ A. Nouri, T.J. Walsh, \& M.L. Littman) \tabrowsep \\
%2006 & USA & Best Student Poster Award, New York Academy of Sciences \tabrowsep \\
%& 2007 & USA & Student-Author Travel Grant, ICML'2007 \\
%& 2008 & USA & Student-Author Travel Grant, ISAIM'2008 \\
%& 2008 & USA & Student-Author Travel Award, ICML'2008 \\
%2008 & USA & {Best Student Paper Award, ICML} \tabrowsep \\
%2008 & USA & Google Student Award, New York Academy of Sciences \tabrowsep \\
%2011 & USA & {Best Paper Award, WSDM} \tabrowsep \\
%2011 & USA & {Notable Paper Award, AISTATS} \tabrowsep \\
%2011 & USA & {Yahoo! Super Star Team Award} (highest team achievement award in the company)
2011 & USA & {Yahoo! Super Star Team Award} (highest team achievement award in the company) \tabrowsep \\
2011 & USA & {Notable Paper Award, AISTATS} \tabrowsep \\
2011 & USA & {Best Paper Award, WSDM} \tabrowsep \\
2008 & USA & {Best Student Paper Award, ICML} \tabrowsep \\
2004 & Canada & {Teaching Assistant Award, University of Alberta}
\end{tabular}
\end{center}

\subsection*{\textsc{\underline{Teaching/Advising Experience}}}

\begin{tabular}{p{30mm} p{120mm}}
Summers since 2013 & Supervised student interns at Microsoft Research and at Google \\
                   & {\small\textmd{Projects on reinforcement learning, multi-armed bandits, imitation learning and Web search}} \\
%Spring 2014 	   & PhD dissertation defense committee (Olivier Nicol), INRIA, France \\
Summers 2010/2011      & Supervised student interns at Yahoo! Labs \\
                   & {\small\textmd{Projects on anomaly detection in distributed file systems, large-scale prediction models in advertising, and news ranking}} \\
Spring 2009 & Guest lecturer for a graduate-level course at the Rutgers University \\
                   & {\small\textmd{Taught the least-squares policy iteration (LSPI) algorithm in the course ``Learning and Sequential Decision Making''.}} \\
09/2007 -- 12/2007 & Co-organizer for a graduate seminar at the Rutgers University \\
                   & {\small\textmd{Compiled reading materials, arranged weekly meetings, and presented papers for ``Planning in Learned Environments'' (w/ Michael Littman).}}\tabrowsep \\
05/2005 -- 08/2005 & Organizer for a graduate seminar at the Rutgers University \\
                   & {\small\textmd{Compiled reading materials, arranged weekly meetings, presented papers, and invited an external speaker for ``Abstractions and Hierarchies for Learning and Planning''.}}\tabrowsep \\
09/2002 -- 07/2004 & Teaching Assistant at the University of Alberta \\
                   & {\small\textmd{Taught seminar sessions and graded homework for the undergraduate course on discrete mathematics: ``Formal Systems and Logic in Computing Science''.}} %\tabrowsep \\
%05/2002 -- 05/2002 & Tutor at a training center \\
%                   & {\small\textmd{Taught Microsoft Visual Basic}}
\end{tabular}

\subsection*{\textsc{\underline{Professional Activities}}}

\begin{compactitem}

\item{Conference Organization} \negitemspace

\begin{compactitem}

\item{Area Chair and/or Senior Program Committee Member}

\begin{compactitem}

\item{AAAI Conference on Artificial Intelligence (AAAI): 2017, 2018}

\item{International Conference on Artificial Intelligence and Statistic (AISTATS): 2017}

\item{International Conferences on Machine Learning (ICML): 2012--2017}

\item{International Joint Conferences on Artificial Intelligence (IJCAI): 2011, 2016, 2017}

\item{Annual Conferences on Neural Information Processing Systems (NIPS): 2014, 2017}

\end{compactitem}

\item{Workshop Co-chairs}

\begin{compactitem}

\item{Reinforcement Learning Competition
(ICML/UAI/COLT'09 Workshop)}

\item{PASCAL2 Exploration \& Exploitation Challenge (ICML'12 Workshop)}

\item{Large-Scale Online Learning and Decision-Making Workshop (Cumberland Lodge, 2012)}

\item{IEEE BigData Workshop (DC, USA, 2014)}

\item{WWW Workshop on Offline and Online Evaluation of Web-based Services (Florence, Italy, 2015)}

\item{SIAM Conference on Optimization --- Algorithms for Reinforcement Learning Minisymposium (Vancouver, Canada, 2017)}

\item{AI Frontiers (San Jose, CA, USA, November 2017)}

\item{From “What If“ to “What Next” (NIPS’17 Workshop)}

\end{compactitem}

\item{Workshop program committee member}

\begin{compactitem}

\item{Planning and Learning in A Priori Unknown or Dynamic Domains, IJCAI 2005}

\item{Abstraction in Reinforcement Learning, ICML/UAI/COLT 2009}

\item{Bayesian Optimization, Experimental Design and Bandits, NIPS, 2011}

\item{AdML: Online Advertising Workshop, ICML 2012}

\item{Bayesian Optimization \& Decision Making, NIPS 2012}

\end{compactitem}

\end{compactitem} \negitemspace

\item{Tutorials} \negitemspace

\begin{compactitem}

\item{``Offline Evaluation and Optimization for Interactive Systems: A Practical Guide'', at the \venuefont{8th International Conference on Web Search and Data Mining (WSDM)}, Shanghai, China, February, 2015.}

\item{``Neural Approaches to to Conversational AI'', with Jianfeng Gao and Michel Galley, at the 56th
Annual Meeting of the Association for Computational Linguistics (ACL), Melbourne, Australia, July, 2018.}

\end{compactitem} \negitemspace

\item{Referee for funding agencies} \negitemspace

\begin{compactitem}

\item{Natural Sciences and Engineering Research Council of Canada (NSERC)}

\item{United States-Israel Binational Science Foundation (BFS)}

\end{compactitem}

\item{Referee for journals} \negitemspace

\begin{compactitem}

\item{ACM Transactions on Intelligent Systems and Technology}

\item{ACM Transactions on Knowledge Discovery from Data}

\item{Advances in Complex Systems}

\item{Artificial Intelligence}

\item{Artificial Intelligence Communications}

\item{Computer Speech and Language}

\item{Data Mining and Knowledge Discovery}

\item{IEEE Journal of Selected Topics in Signal Processing}

\item{IEEE Transactions on Automatic Control}

\item{IEEE Transactions on Knowledge and Data Engineering}

\item{IEEE Transactions on Neural Networks}

\item{IEEE Transactions on Wireless Communications}

\item{Journal of Artificial Intelligence Research}

\item{Journal of Computer Science and Technology}

\item{Journal of Machine Learning Research}

\item{Journal of Selected Topics in Signal Processing}

\item{Machine Learning}

\item{Mathematics of Operations Research}

\item{Neural Computation}

\item{Neurocomputing}

\end{compactitem} \negitemspace

\item{Referee for conferences (including services as area chair and senior program committee member):}

\begin{compactitem}

\item{AAAI (AAAI Conferences on Artificial Intelligence): 2006, 2008, 2010, 2016 (Demo), 2017 (SPC)}

\item{AISTATS (International Conferences on Artificial Intelligence and Statistics): 2011, 2017 (SPC)}

\item{ALT (International Conferences on Algorithmic Learning Theory): 2015}

\item{COLT (Annual Conferences on Learning Theory): 2010, 2011, 2012, 2015}

\item{ECML (European Conferences on Machine Learning): 2009}

\item{KDD (ACM SIGKDD Conferences on Knowledge Discovery and Data Mining): 2012}

\item{ICML (International Conferences on Machine Learning): 2009--2011, 2012--2017 (AC)}

\item{IJCAI (International Joint Conferences on Artificial Intelligence): 2007, 2011 (SPC), 2015, 2016 (SPC)}

\item{NIPS (Annual Meetings on Neural Information Processing Systems): 2008--2013, 2014 (AC)}

\item{STOC (ACM Symposium on Theory of Computing): 2014}

\item{UAI (Annual Conferences on Uncertainty in Artificial
Intelligence): 2010, 2012, 2016}

\item{UbiComp (International Conferences on Ubiquitous Computing): 2011}

\item{WSDM (ACM International Conferences on Web Search and Data Mining): 2012, 2013}

\item{WWW (International Conferences on World Wide Web): 2012}

\end{compactitem} \negitemspace

%\item{Conference presentations (oral/spotlight/poster) at:
%IJCAI'03, NIPS'03 (workshop), ECML'04, AAAI'05, ICML'06,
%AI\&Math'08, ICML'08, NIPS'08, Asilomar SSC'09, NIPS'09 (workshop), WWW'10}
%\negitemspace

\item{Open source and dataset contributions} \negitemspace

\begin{compactitem}

\item{Vowpal Wabbit: an open source project started with John Langford and
Alexander L. Strehl for fast online learning in large-scale
prediction problems.  URL: http://www.hunch.net/\~{}vw~}

\item{Yahoo! Front Page Today Module User Click Log Dataset: the first large-scale real-life dataset that supports unbiased evaluation of multi-armed bandit algorithms (with help from Wei Chu).  \\ URL: http://webscope.sandbox.yahoo.com/catalog.php?datatype=r}

\item{Deep reinforcement learning package in Microsoft CNTK (with Yi Mao et al.)}

\end{compactitem} \negitemspace

\end{compactitem}

\subsection*{\textsc{\underline{Invited Talks}}}

\begin{compactitem}

\item{Primal-dual Approaches to Reinforcement Learning}
\begin{compactitem}
\item{International Symposium on Mathematical Programming (ISMP), Bordeaux, France. July, 2018. (scheduled)}
\item{INFORMS International Conference, Taipei, Taiwan. June, 2018. (scheduled)}
\item{Annual Conference on Information Sciences and Systems (CISS), Princeton, NJ, USA. March, 2018.}
\item{Google Machine Learning Day, Beijing, China. March, 2018.}
\item{Department of Electrical Engineering, Stanford University, Palo Alto, CA, USA. February, 2018.}
\item{Google Brain, Montreal, QC, Canada.  September, 2017.}
\item{New York University, New York, NY, USA.  May, 2017.}
\item{Simons Institute, Berkeley, CA, USA.  February, 2017.}
\end{compactitem}

\item{Reinforcement Learning for Conversational Systems}
\begin{compactitem}
\item{Google Brain, Montreal, QC, Canada.  September, 2017.}
\item{ICML Workshop on Interactive Machine Learning and Semantic Information Retrieval, Sydney, AU.  August, 2017.}
\item{Multidisciplinary Conference on Reinforcement Learning and Decision Making (RLDM), Ann Arbor, MI, USA.  June, 2017.}
\item{Korea Advanced Institute of Science and Technology, Korea. June 2017.}
\item{Sungkyunkwan University, Suwon, Korea. June 2017.}
\item{ACML Workshop on Reinforcement Learning, Hamilton, NZ.  November, 2016.}
\item{Global AI Conference, Shanghai, China.  November, 2016.}
\end{compactitem}

\item{Off-policy Learning and Counterfactual Evaluation}
\begin{compactitem}
\item{Graduate School of Business, Stanford University, CA, USA.  May, 2017.}
\item{Oxford University, Oxford, UK. November, 2015.}
\item{Google DeepMind, London, UK. November, 2015.}
\item{AdTech LA Meetup, Santa Monica, CA, USA. October, 2015.}
\item{UW CSE MSR Summer Institute, Union, WA, USA. August, 2015.}
\item{INRIA SequeL, Lille, France. December, 2014.}
\item{Criteo, Paris, France. December, 2014.}
\item{Department of Computing Science, University of Alberta, Edmonton, AB, Canada.  November, 2014.}
\item{KDD Workshop on User Engagement Optimization, New York, NY, USA.  August, 2014.}
\item{AAAI Workshop on Sequential Decision-Making with Big Data, Queb{\'e}c City, QC, Canada. July, 2014.}
\item{Microsoft Research Latin American Faculty Summit, Vi\~na del Mar, Chile. May, 2014.}
\item{IEEE Information Theory and Application (ITA) Workshop, San Diego, CA, USA.  February, 2014.}
\item{Distinguished Faculty and Graduate Student Seminar, Department of Statistics, University of Michigan, Ann Arbor, MI, USA.  February, 2014.}
\end{compactitem}

\item{Machine Learning in the Bandit Setting: Algorithms, Evaluation, and Case Studies}
\begin{compactitem}
\item{Department of Computer Science, University of South California, Los Angeles, CA, USA. October, 2015.}
\item{Department of Computer Science, Purdue University, West Lafayette, IN, USA. April, 2014.}
\item{Joint Statistical Meetings (Statistics in Marketing Track), Montreal, QC, Canada.  August, 2013.}
\item{Tenth National Symposium of Search Engine and Web Mining, Beijing, China.  May 2012.}
\item{Microsoft Research Asia, Beijing, China.  May 2012.}
\item{Department of Machine Intelligence, Peking University, Beijing, China.  May 2012.}
\item{Department of Computer Science and Technology, Tsinghua University, Beijing, China.  May 2012.}
\item{Department of Computer Science and Engineering, University of California, Los Angeles, CA, USA.  May 2012.}
\item{Department of Computer Science and Engineering, University of California, San Diego, CA, USA.  May 2012.}
\item{Department of Computer Science, University of California, Irvine, CA, USA.  May 2012.}
\item{Google Research, Mountain View, CA, USA. April 2012.}
\item{Microsoft Research, Redmond, WA, USA. April 2012.}
\item{Adobe Advanced Technology Labs, San Jose, CA, USA.  April 2012.}
\item{Microsoft Research, Mountain View, CA, USA. April 2012.}
\item{Department of Computer Science, Virginia Tech, Blacksburg, VA, USA.  February 2012.}
\item{Department of Computer Science, Johns Hopkins University, MD, USA.  February 2012.}
\item{Technicolor Research Center, Palo Alto, CA, USA.  February 2012.}
\item{Department of Computing Science, University of Alberta, Edmonton, AB, Canada. June 2011.}
\item{Industrial Affiliates Annual Conference, Department of Statistics, Stanford University, USA.  May 2011.  With Deepak Agarwal and Bee-Chung Chen.}
\item{Microsoft Sillicon Valley Center, Mountain View, CA, USA.  March 2011.}
\item{Artificial Intelligence Center, SRI International, Menlo Park, CA, USA.  April 2010.}
\end{compactitem} \negitemspace

\item{``Vowpal Wabbit for Extremely Fast Machine Learning''}
\begin{compactitem}
\item{GraphLab Workshop on Big Learning, San Francisco, CA, USA.  July, 2012.}
\item{First data mining meetup on large-scale machine learning algorithms, San Francisco, CA, USA.  August 2011.}
\end{compactitem} \negitemspace

\item{``A Unifying Framework for Computational Reinforcement
Learning Theory''}
\begin{compactitem}
\item{ICML Workshop on Planning and Acting with Uncertain Models, Bellevue, WA, USA.  June 2011.}
\item{Department of Computing Science, University of Alberta, Edmonton, AB, Canada. June 2011.}
\item{Yahoo! Research, Sunnyvale, CA, USA.  April 2009.}
\item{Google Research, New York, NY, USA.  April 2009.}
\item{Yahoo! Research, New York, NY, USA.  January 2009.}
\item{Reasoning and Learning Laboratory, McGill University,
McGill, QC, Canada.  May 2008.}
\item{DARPA Information Processing Technology meeting, Arlington, VA, USA.  February 2008.}
\item{AT\&T Shannon Labs, Florham Park, NJ, USA.  January 2008.}
\end{compactitem} \negitemspace

\item{``Sparse Online Learning via Truncated Gradient''}
\begin{compactitem}
\item{Asilomar Conference on Signals, Systems, and Computers,
Pacific Grove, CA, USA.  November 2009.}
\item{eBay Research Labs, San Jose, CA, USA.  April 2009.}
\item{Department of Information Analysis \& Management, NEC
Laboratories America, Cupertino, CA, USA.  April 2009.}
\item{Text Analysis and Machine Learning Group, University of
Ottawa, Ottawa, ON, Canada.  May 2008.}
\end{compactitem} \negitemspace

\end{compactitem}

\subsection*{\textsc{\underline{Publications}}
%\footnote{Authors are ordered alphabetically in \emph{some} papers.}
}

\paragraph{Journal Papers} \negitemspace

\begin{compactenum}[(J1)]

\item{M. Dud\'ik, D. Erhan, J. Langford, and \myself: Doubly robust policy evaluation and optimization.  In \venuefont{Statistical Science}, 29(4):485--511, 2014.}

\item{J. Bian, B. Long, \myself, T. Moon, A. Dong, and Y. Chang: Exploiting user preference for online learning in Web content optimization systems.  In \venuefont{ACM Transactions on Intelligent Systems and Technology}, 5(2), 2014.}

\item{T. Moon, W. Chu, \myself, Z. Zheng, and Y. Chang: Refining recency search results with user click feedback.  In \venuefont{ACM Transactions on Information Systems}, 30(4), 2012.}

\item{J. Langford, \myself, P. McAfee, and K. Papineni: Cloud control: Voluntary admission control for Intranet traffic management.  In \venuefont{Information Systems and e-Business Management}, 10(3):295--308, 2012.}

\item{\myself, M.L. Littman, T.J. Walsh, and A.L. Strehl:
Knows what it knows: A framework for self-aware learning.  In \venuefont{Machine Learning}, 82(3):399--443, 2011.}

\item{\myself\ and M.L. Littman: Reducing reinforcement learning
to KWIK online regression.  In the \venuefont{Annals of Mathematics and
Artificial Intelligence}, 58(3--4):217--237, 2010.}

\item{J. Langford, \myself, J. Wortman, and Y.
Vorobeychik: Maintaining equilibria during exploration in
sponsored search auctions.  In \venuefont{Algorithmica},
58(4):990--1021, 2010.}

\item{A.L. Strehl, \myself, and M.L. Littman:
Reinforcement learning in finite MDPs: PAC analysis.  In the
\venuefont{Journal of Machine Learning Research}, 10:2413--2444, 2009.}

\item{E. Brunskill, B.R. Leffler, \myself, M.L. Littman,
and N. Roy: Provably efficient learning with typed parametric
models.  In the \venuefont{Journal of Machine Learning Research},
10:1955--1988, 2009.}

\item{J. Langford, \myself, and T. Zhang: Sparse online
learning via truncated gradient. In the \venuefont{Journal of Machine
Learning Research}, 10:777--801, 2009.}

\item{T.J. Walsh, A. Nouri, \myself, and M.L. Littman:
Planning and learning in environments with delayed feedback. In
the \venuefont{Journal of Autonomous Agents and Multi-Agent
Systems}, 18(1):83--105, 2009.}

\item{\myself, V. Bulitko, and R. Greiner: Focus of
attention in reinforcement learning. In the \venuefont{Journal of
Universal Computer Science}, 13(9):1246--1269, 2007.}

\item{\myself, M. Shao, Z. Zheng, C. He, and Z.-H. Du:
Typical XML document transformation methods and an application
system (in Chinese). \venuefont{Computer Science}, 30(2):40--44,
February, 2003.}

\end{compactenum}

\paragraph{Refereed Conference Papers} \negitemspace

\begin{compactenum}[(C1)]

\item{B. Dai, A. Shaw, N. He, \myself, and L. Song: Boosting the actor with dual critic. In \venuefont{the 6th International Conference on Learning Representations (ICLR)}, 2018.}

\item{Z. Lipton, X. Li, J. Gao, \myself, F. Ahmed, and L. Deng: Efficient dialogue policy learning with BBQ-networks.  In \venuefont{the 32nd AAAI Conference on Artificial Intelligence (AAAI)}, 2018.}

\item{J. Chen, C. Wang, L. Xiao, J. He, \myself, and L. Deng: Q-LDA: Uncovering latent patterns in text-based sequential decision processes. In \venuefont{Advances in Neural Information Processing Systems 30 (NIPS)}, 2017.}

\item{B. Peng, X. Li, \myself, J. Gao, A. Celikyilmaz, S. Lee, K.-F. Wong: Composite task-completion dialogue system via hierarchical deep reinforcement learning.  In \venuefont{the 2017 Conference on Empirical Methods in Natural Language Processing (EMNLP)}, 2017.}

\item{\myself, Y. Lu, and D. Zhou: Provably optimal algorithms for generalized linear contextual bandits.  In \venuefont{the 34th International Conference on Machine Learning (ICML)}, 2017.}

\item{S. Du, J. Chen, \myself, L. Xiao, and D. Zhou: Stochastic variance reduction methods for policy evaluation.  In \venuefont{the 34th International Conference on Machine Learning (ICML)}, 2017.}

\item{B. Dhingra, \myself, X. Li, J. Gao, Y.-N. Chen, F. Ahmed, and L. Deng: Towards end-to-end reinforcement learning of dialogue agents for information access.  In \venuefont{the 55th Annual Meeting of the Association for Computational Linguistics (ACL)}, 2017.}

\item{E. Parisotto, A. Mohamed, R. Singh, \myself, D. Zhou, and P. Kohli: Neuro-symbolic program synthesis.  In \venuefont{the 5th International Conference on Learning Representations (ICLR)}, 2017.}

\item{X. Li, Y.-N. Chen, \myself, J. Gao, A. Çelikyilmaz: End-to-End task-completion neural dialogue systems.  In \venuefont{the ???? (IJCNLP)}, 2017.}

\item{T.K. Huang, \myself, A. Vartanian, S. Amershi, and J. Zhu: Active learning with oracle epiphany.  In Advances in Neural Information Processing Systems 29 (NIPS), 2016.}

\item{J. He, M. Ostendorf, X. He, J. Chen, J. Gao, \myself, and L. Deng: Deep reinforcement learning with a combinatorial action space for predicting and tracking popular discussion threads.  In \venuefont{the 2016 Conference on Empirical Methods in Natural Language Processing (EMNLP)}, 2016.}

\item{C.-Y. Liu and \myself: On the Prior Sensitivity of Thompson Sampling. In \venuefont{the 27th International Conference on Algorithmic Learning Theory (ALT)}, 2016.}

\item{J. He, J. Chen, X. He, J. Gao, \myself, L. Deng, and M. Ostendorf: Deep reinforcement learning with a natural language action space.  In \venuefont{the 54th Annual Meeting of the Association for Computational Linguistics (ACL)}, 2016.}

\item{N. Jiang and \myself: Doubly robust off-policy value evaluation for reinforcement learning.  In \venuefont{the 33rd International Conference on Machine Learning (ICML)}, 2016.}

\item{S. Agrawal, N. R. Devanur, and \myself: An efficient algorithm for contextual bandits with knapsacks, and an extension to concave objectives.  In \venuefont{the 29th Annual Conference on Learning Theory (COLT)}, 2016.}

\item{M. Zoghi, T. Tunys, \myself, D. Jose, J. Chen, C.-M. Chin, and M. de Rijke: Click-based hot fixes for underperforming torso queries.  In \venuefont{the 39th International ACM SIGIR Conference on Research and Development in Information Retrieval (SIGIR)}, 2016.}

\item{J. He, J. Chen, X. He, J. Gao, \myself, L. Deng, and M. Ostendorf: Deep reinforcement learning with an unbounded action space.  In \venuefont{the International Conference on Learning Representations (ICLR), Workshop Track}, 2016.}

\item{\myself, R. Munos, and Cs. Szepesv\'{a}ri: Toward minimax off-policy value estimation.  In \venuefont{the 18th International Conference on Artificial Intelligence and Statistics (AISTATS)}, 2015.}

\item{\myself, S. Chen, J. Kleban, and A. Gupta: Counterfactual estimation and optimization of click metrics in search engines: A case study.  In \venuefont{the 24th International Conference on World Wide Web (WWW), Companion}, 2015.}

\item{\myself, J. Kim, and I. Zitouni: Toward predicting the outcome of an {A/B} experiment for search relevance.  In \venuefont{the 8th International Conference on Web Search and Data Mining (WSDM)}, 2015.}

\item{\myself, H. He, and J.D. Williams: Temporal supervised learning for inferring a dialog policy from example conversations.  In the \venuefont{IEEE Spoken Language Technology Workshop (SLT)}, 2014.}

\item{A. Agarwal, D. Hsu, S. Kale, J. Langford, \myself, and R.E. Schapire: Taming the monster: A fast and simple algorithm for contextual bandits.  In \venuefont{the 31st International Conference on Machine Learning (ICML)}, 2014.}

\item{E. Brunskill and \myself: PAC-inspired option discovery in lifelong reinforcement learning.  In \venuefont{the 31st International Conference on Machine Learning (ICML)}, 2014.}

\item{E. Brunskill and \myself: Sample complexity of multi-task reinforcement learning.  In \venuefont{the 29th Conference on Uncertainty in Artificial Intelligence (UAI)}, 2013.}

\item{M. Dud\'ik, D. Erhan, J. Langford, and \myself: Sample-efficient nonstationary-policy evaluation for contextual bandits.  In \venuefont{the 28th Conference on Uncertainty in Artificial Intelligence (UAI)}, 2012.}

\item{\myself, W. Chu, J. Langford, T. Moon, and X. Wang: An unbiased offline evaluation of contextual bandit algorithms with generalized linear models.  In \venuefont{Journal of Machine Learning Research - Workshop and Conference Proceedings 26: On-line Trading of Exploration and Exploitation 2}, 2012.}

\item{V. Navalpakkam, R. Kumar, \myself, and D. Sivakumar: Attention and selection in online choice tasks.  In \venuefont{the 20th International Conference on User Modeling, Adaptation and Personalization (UMAP)}, 2012}

\item{H. Wang, A. Dong, \myself, Y. Chang, and E. Gabrilovich: Joint relevance and freshness learning From clickthroughs for news search.  In \venuefont{the 21st International Conference on World Wide Web (WWW)}, 2012.}

\item{O. Chapelle and \myself: An empirical evaluation of Thompson sampling.  In \venuefont{Advances in Neural Information Processing Systems 24 (NIPS)}, 2012.}

\item{M. Dud\'ik, J. Langford, and \myself: Doubly robust policy evaluation and learning.  In \venuefont{the 28th International Conference on Machine
Learning (ICML)}, 2011.}

\item{W. Chu, M. Zinkevich, \myself, A. Thomas, and B. Tseng: Unbiased online active learning in data streams.  In \venuefont{the 17th ACM SIGKDD Conference on Knowledge Discovery and Data Mining (KDD)}, 2011.}

\item{D. Agarwal, \myself, and A.J. Smola: Linear-time algorithms for propensity scores.  In \venuefont{the 14th International Conference on Artificial Intelligence and Statistics (AISTATS)}, 2011.}

\item{A. Beygelzimer, J. Langford, \myself, L. Reyzin, and R.E. Schapire: Contextual bandit algorithms with supervised learning guarantees.  In \venuefont{the 14th International Conference on Artificial Intelligence and Statistics (AISTATS)}, 2011.  \textbf{Co-winner of the Notable Paper Award.}}

\item{W. Chu, \myself, L. Reyzin, and R. Schapire: Linear contextual bandit problems.  In \venuefont{the 14th International Conference on Artificial Intelligence and Statistics (AISTATS)}, 2011.}

\item{\myself, Wei Chu, John Langford, and Xuanhui Wang: Unbiased offline evaluation of contextual-bandit-based news article recommendation algorithms.  In \venuefont{the 4th ACM International Conference on Web Search and Data Mining (WSDM)}, 2011.  \textbf{Winner of the Best Paper Award.}}

\item{A.L. Strehl, J. Langford, \myself, and S. Kakade: Learning from logged implicit exploration data.  In \venuefont{Advances in Neural Information Processing Systems 23 (NIPS)}, 2011.}

\item{M. Zinkevich, M. Weimer, A.J. Smola, and \myself: Convergence rates of parallel online learning via stochastic gradient descent.  In \venuefont{Advances in Neural Information Processing Systems 23 (NIPS)}, 2011.}

\item{T. Moon, \myself, W. Chu, C. Liao, Z. Zheng, and Y. Chang:  Online learning for recency search ranking using real-time user feedback (short paper).  In \venuefont{the 19th ACM Conference on Information and Knowledge Management (CIKM)}, 2010.}

\item{\myself, W. Chu, J. Langford, and R.E. Schapire: A contextual-bandit approach to personalized news article recommendation.  In \venuefont{the 19th International Conference on World Wide Web (WWW)}, 2010.}

\item{Y. Xie, Y. Zhang, and \myself: Neuro-fuzzy reinforcement learning for adaptive intersection traffic signal control.  In \venuefont{the Annual Meeting of Transportation Research Board (TRB)}, 2010.}

\item{\myself, J.D. Williams, and S. Balakrishnan:
Reinforcement learning for spoken dialog management using
least-squares policy iteration and fast feature selection. In
\venuefont{the 10th Annual Conference of the International Speech
Communication Association (INTERSPEECH)}, 2009.}

\item{C. Diuk, \myself, and B.R. Leffler: The adaptive
$k$-meteorologists problem and its application to structure
learning and feature selection in reinforcement learning. In
\venuefont{the 26th International Conference on Machine
Learning (ICML)}, 2009.}

\item{J. Asmuth, \myself, M.L. Littman, A. Nouri, and D.
Wingate: A Bayesian sampling approach to exploration in
reinforcement learning.  In \venuefont{the 25th
International Conference on Uncertainty in Artificial Intelligence
(UAI)}, 2009.}

\item{\myself, M.L. Littman and C.R. Mansley: Online
exploration in least-squares policy iteration. In \venuefont{the
8th International Conference on Autonomous Agents and
Multiagent Systems (AAMAS)}, 2009}.

\item{L. Langford, \myself, and T. Zhang: Sparse online
learning via truncated gradient.  In \venuefont{Advances in Neural
Information Processing Systems 21 (NIPS)}, 2009.}

\item{\myself: A worst-case comparison between temporal
difference and residual gradient.  In \venuefont{the 25th
International Conference on Machine Learning (ICML)}, 2008.}

\item{\myself, M.L. Littman, and T.J. Walsh: Knows what it
knows: A framework for self-aware learning. In \venuefont{the
25th International Conference on Machine Learning (ICML)},
2008.  \textbf{Co-winner of the Best Student Paper Award.
A Google Student Award winner at the New York Academy of Sciences
Symposium on Machine Learning, 2008.}}

\item{R. Parr, \myself, G. Taylor, C. Painter-Wakefield, and
M.L. Littman: An analysis of linear models, linear value function
approximation, and feature selection for reinforcement learning.
In \venuefont{the 25th International Conference on Machine
Learning (ICML)}, 2008.}

\item{E. Brunskill, B.R. Leffler, \myself, M.L. Littman, and N.
Roy: CORL: A continuous-state offset-dynamics reinforcement
learner.  In \venuefont{the 24th Conference on Uncertainty in
Artificial Intelligence (UAI)}, 2008.}

\item{\myself\ and M.L. Littman: Efficient value-function
approximation via online linear regression.  In \venuefont{the 10th
International Symposium on Artificial Intelligence and Mathematics
(AI\&Math)}, 2008.}

\item{J. Wortman, Y. Vorobeychik, \myself, and J. Langford:
Maintaining equilibria during exploration in sponsored search
auctions.  In \venuefont{the 3rd International Workshop on Internet
and Network Economics (WINE)}, LNCS 4858, 2007.}

\item{T.J. Walsh, A. Nouri, \myself, and M.L. Littman:
Planning and learning in environments with delayed feedback.  In
\venuefont{the 18th European Conference on Machine Learning
(ECML)}, LNCS 4701, 2007.}

\item{R. Parr, C. Painter-Wakefield, \myself, and M.L.
Littman: Analyzing feature generation for value-function
approximation.  In \venuefont{the 24th International
Conference on Machine Learning (ICML)}, 2007.}

\item{A.L. Strehl, \myself, E. Wiewiora, J. Langford, and M.L. Littman: PAC model-free reinforcement learning.  In \venuefont{the 23rd International Conference on Machine Learning (ICML)}, 2006.  \textbf{Best Student Poster Award winner at the New York Academy of Sciences Symposium on Machine Learning, 2006.}}

\item{A.L. Strehl, \myself, and M.L. Littman: Incremental model-based learners with formal learning-time guarantees.  In \venuefont{the 22nd Conference on Uncertainty in Artificial Intelligence (UAI)}, 2006.}

\item{\myself, T.J. Walsh, and M.L. Littman: Towards a unified theory of state abstraction for MDPs. In \venuefont{the 9th International Symposium on Artificial Intelligence and Mathematics (AI\&Math)}, 2006.}

\item{\myself, M.L. Littman: Lazy approximation for solving continuous finite-horizon MDPs. In \venuefont{the 20th National Conference on Artificial Intelligence (AAAI)}, 2005.}

\item{\myself, V. Bulitko, and R. Greiner: Batch reinforcement learning with state importance (extended abstract).  In \venuefont{the 15th European Conference on Machine Learning (ECML)}, LNCS 3201, 2004.}

\item{V. Bulitko, \myself, R. Greiner, and I. Levner: Lookahead pathologies for single agent search (poster paper). In \venuefont{the 18th International Joint Conference on Artificial Intelligence (IJCAI)}, 2003.}

\item{I. Levner, V. Bulitko, \myself, G. Lee, and R. Greiner: Towards automated creation of image interpretation systems. In \venuefont{the 16th Australian Joint Conference on Artificial
Intelligence}, LNCS 2903, 2003.}

\item{\myself, V. Bulitko, R. Greiner, and I. Levner: Improving an adaptive image interpretation system by leveraging. In \venuefont{the 8th Australian and New Zealand Intelligent Information System Conference}, 2003.}

\end{compactenum} \negitemspace

\paragraph{Book and Book Chapters} \negitemspace

\begin{compactenum}[(B1)]

\item{K. Hofmann, \myself, and F. Radlinski: Online Evaluation for Information Retrieval.  Foundations and Trends in Information Retrieval, 10(1):1--107, 2016. ISBN 978-1-68083-163-4.}

\item{\myself: Sample complexity bounds of exploration.  In Marco Wiering and Martijn van Otterlo, editors, \venuefont{Reinforcement Learning: State of the Art}, Springer Verlag, 2012.}

\item{M. Shao, \myself, Z. Zheng, and C. He: Practical
Programming in XML. \venuefont{Tsinghua University Press}, Beijing,
China, December, 2002. ISBN 7-900643-85-0.}

\end{compactenum} \negitemspace

\paragraph{Theses} \negitemspace

\begin{compactenum}[(T1)]

\item{\myself: A unifying framework for computational
reinforcement learning theory. \venuefont{Doctoral dissertation},
Department of Computer Science, Rutgers University, New Brunswick,
NJ, USA, May, 2009.}

\item{\myself: Focus of attention in reinforcement learning.
\venuefont{MSc thesis}, Department of Computing Science, University
of Alberta, Edmonton, Alberta, Canada, July, 2004.}

\item{\myself: Design and implementation of an agent
communication module based on KQML. \venuefont{Bachelor degree
thesis}, Department of Computer Science and Technology, Tsinghua
University, Beijing, China, June, 2002.}

\end{compactenum} \negitemspace

\paragraph{Other Papers} \negitemspace

\begin{compactenum}[(O1)]

\item{X. Li, Z.C. Lipton, B. Dhingra, \myself, J. Gao, Y.-N. Chen: A user simulator for task-completion dialogues.  MSR technical report, December 2016.}

\item{E. Brunskill and \myself: The online discovery problem and its application to lifelong reinforcement learning.  CoRR abs/1506.03379, June 2015.}

\item{D. Yankov, P. Berkhin, and \myself: Evaluation of explore-exploit policies in multi-result ranking systems.  Microsoft Journal on Applied Research, volume 3, pages 54--60, 2015.  Also available as Microsoft Research Technical Report \#MSR-TR-2015-34, May 2015.}

\item{Z. Qin, V. Petricek, N. Karampatziakis, \myself, and J. Langford: Efficient online bootstrapping for large scale learning.  \venuefont{NIPS Workshop on Big Data}, December, 2013.  Also available as Microsoft Research Technical Report \#MSR-TR-2013-132.}

\item{\myself\ and O. Chapelle: Regret bounds for Thompson sampling (Open Problems).  In the \venuefont{Twenty-Fifth Annual Conference on Learning Theory (COLT)}, 2012}

\item{\myself\ and M.L. Littman: Prioritized sweeping converges to the optimal value function.  Technical report DCS-TR-631, Department of Computer Science, Rutgers University, May 2008.}

\item{A.L. Strehl, \myself, and M.L. Littman: PAC reinforcement learning bounds for RTDP and Rand-RTDP.  \venuefont{AAAI technical report WS-06-11}, pages 50-56, July 2006.}

\item{\myself\ and M.L. Littman: Lazy approximation: A new approach for solving continuous finite-horizon MDPs.  Technical report DCS-TR-577, Department of Computer Science, Rutgers University, May 2005.}

\item{\myself, V. Bulitko, and R. Greiner: Focus of attention in sequential decision making. AAAI technical report WS-04-08, pages 43-48, July 2004.}

\end{compactenum} \negitemspace

\iffalse
\paragraph{Working Papers (Under Review/In Preparation)} \negitemspace

\begin{compactenum}[(W1)]

%\item{D. Agarwal, K Borgwardt, A. Gretton, J. Huang, Q. Le, \myself, M. Schmittfull, B. Sch\"{o}lkopf, and A.J. Smola: Covariate shift and local learning by distribution matching.}%  In preparation.}

%\item{D. Donato, R. Greiner, \myself, G. Theocharous, and E. Velipasaoglu: A contextualized search assistant based on user browsing history.}%  In preparation.}

%\item{R. Greiner, \myself, and B. Long: Cross-domain active learning.}%  In preparation.}

%\item{R. Kuma, \myself, V. Navalpakkam, and D. Sivakumar: Modeling attention in multi-item choice tasks.}%  Under review by the Proceedings of the National Academy of Sciences (PNAS).}

%\item{G. Kathalagiri, \myself, A. Smola, and A. Youssefi: Detecting anomalies in the Hadoop distributed file system. In preparation.}

%\item{\myself\ and M.L. Littman: Prioritized sweeping converges to the optimal value function.  Technical report DCS-TR-631, Department of Computer Science, Rutgers University, May 2008.  To be submitted to the \venuefont{Machine Learning} journal.}

\item{M. Dud\'ik, D. Erhan, J. Langford, and \myself: Doubly robust policy evaluation and learning (journal version).}

\item{\myself: Generalized Thompson sampling for contextual bandits.  In preparation.}

\item{\myself, S. Chen, A. Gupta, and J. Kleban: Counterfactual analysis of click metrics for search engine
optimization.  Submitted.}

\item{\myself, H. He, and J.D. Williams: Temporal supervised learning: A method for batch imitation learning.  Submitted.}

\end{compactenum}

\fi

\iffalse
\subsection*{\textsc{\underline{Patents Pending}}}

\begin{compactitem}

\item{``System and Method for Automatically Generating a Dialog Manager'', with Jason Williams and Suhrid Balakrishnan, filed in December 2009.}

\item{``A Contextual-bandit Approach to Personalized News Article Recommendation'', with Wei Chu, John Langford, and Robert Schapire, filed in July 2009.}

\item{``Unbiased Online Active Learning in UGC Data Streams'', with Wei Chu, Martin Zinkevich, Achint Oommen Thomas, and Belle Tseng, filed in February 2012.}

\item{``Joint Relevance and Freshness Learning From Clickthroughs for News Search'', with Hongning Wang, Anlei Dong, Yi Chang, and Evgeniy Gabrilovich.  In preparation, 2011.}

\item{``Pairwise Learning Methodology for Online Learning in Recommender Systems'', with Jiang Bian, Bo Long, Taesup Moon, Anlei Dong, and Yi Chang.  In preparation, 2011.}

\item{``Counterfactual evaluation and optimization of query reformulation in search engines'', with Shunbao Chen, Ankur Gupta, and Jim Kleban.  In preparation, 2014.}

%\item{Thompson sampling for exploration/exploitation tradeoff in a large-scale display advertising system, with Olivier Chapelle, in preparation with firm.}

\end{compactitem}

\fi

%\subsection*{\textsc{\underline{Selected Applied Research Projects}}}

%\begin{compactitem}

%\item{\textbf{Vowpal Wabbit} (with J. Langford and A. Strehl, 2007): an open-source package for large-scale machine learning that was used by multiple products inside and outside Yahoo! like Y! Mail and eHarmony.}

%\item{\textbf{LinUCB} (with W. Chu, J. Langford, and R. Schapire, 2009): an efficient contextual explore/exploit algorithm that shows competitive results both in personalized article recommendation in COKE and in online re-ranking in MLR.  A patent application has been filed.}

%\item{\textbf{Offline Bandit Policy Evaluation} (with J. Langford, W. Chu, and X. Wang, 2010): an unbiased method for evaluating explore/exploit algorithm directly from online log in random buckets, which shows highly accurate and reliable offline evaluation results.  It avoids frequent use of the expensive bucket tests, and is successfully applied to a few important problems at Yahoo! like content recommendation.  Its application to more problems like advertising is under active investigation.  It won the best paper award at WSDM.}

%\item{\textbf{Efficient Propensity Score Estimation} (with A. Smola and D. Agarwal, 2010): a few new linear-time algorithms designed for solving large-scale propensity score estimation problems that are critical for data analysis at the present of sampling bias (such as in advertising, ranking, etc.).}

%\item{\textbf{Online Learning to Re-rank} (with T. Moon, W. Chu, C. Liao, Z. Zheng, and Y. Chang, 2010): a principled, real-time approach to refining search results from a search engine by employing user click feedback, which shows promises for time-sensitive queries.}

%\item{\textbf{Voluntary Admission Control for Yahoo!'s Backbone Network} (with K. Papineni, J. Langford, and P. McAfee, 2009): a bandwidth throttling method to reduce peak traffic volume of backbone network by adaptive control and reinforcement learning.}

%\item{\textbf{Hadoop Error Detection} (with the Grid Management System team, 2010): a project that uses machine learning to detect system failure (software and/or hardware) in name nodes and job trackers for Hadoop.}

%\item{\textbf{Spoken Dialog Management} (with J. Williams and S. Balakrishnan at AT\&T Shannon Labs, 2008): a method that accelerates a state-of-the-art reinforcement learning algorithm so that it can be applied in realistic spoken dialog management systems.  A patent application has been filed.}

%\end{compactitem}

%\subsection*{\textsc{\underline{Collaborators (Coauthors)}}}
%
%\begin{compactitem} \negitemspace
%\item{\emph{AT\&T Shannon Labs}: Suhrid Balakrishnan, Jason D. Williams}
%\item{\emph{Duke University}: Christopher Painter-Wakefield, Ronald Parr, Gavin Taylor}
%\item{\emph{IBM Research}: Alina Beygelzimer}
%\item{\emph{Massachusetts Institute of Technology}: Emma Brunskill, Nicholas Roy, David Wingate}
%%\item{\emph{Max Planck Institutes}: Karsten Borgwardt, Marcel Schmittfull, Bernhard Sch\"olkopf}
%\item{\emph{Michigan State University}: Tianbao Yang}
%%\item{\emph{Microsoft}: Jiayuan Huang}
%\item{\emph{Princeton University}: Robert E. Schapire}
%\item{\emph{Rutgers University}: John Asmuth, Carlos Diuk, Bethany R. Leffler, Michael L. Littman, Christopher R. Mansley, Ali Nouri, Alexander L. Strehl, Thomas J. Walsh, Tong Zhang}
%%\item{\emph{Stanford University}: Quoc V. Le}
%%\item{\emph{State University of New York at Stony Brook}: Girish Kathalagiri}
%\item{\emph{Texas A\&M University}: Yuanchang Xie, Yunlong Zhang}
%\item{\emph{Tsinghua University}: Zhi-Hui Du, Chuan He, Min Shao, Zhengkun Zheng}
%%\item{\emph{University College London}: Arthur Gretton}
%\item{\emph{University of Alberta}: Vadim Bulitko, Russell Greiner, Greg Lee, Ilya Levner}
%\item{\emph{University of California at San Diego}: Eric Wiewiora}
%\item{\emph{University of Illinois at Urbana Champaign}: Hongning Wang}
%\item{\emph{University of Michigan}: Yevgeniy Vorobeychik}
%\item{\emph{University of Pennsylvania}: Sham Kakade, Jennifer Wortman}
%\item{\emph{Yahoo! Labs}: Deepak Agarwal, Jiang Bian, Yi Chang, Olivier Chapelle, Wei Chu, %Debora Donato,
%Anlei Dong, Miroslav Dud\'ik, Dumitru Erhan, Evgeniy Gabrilovich, Ravi Kuma, John Langford, Ciya Liao, Bo Long, Preston McAfee, Taesup Moon, Kishore Papineni, Lev Reyzin, Vidhya Navalpakkam, Dandapani Sivakumar, Alexander J. Smola, %Georgios Theocharous,
%Achint Thomas, Belle Tseng, %Emre Velipasaoglu,
%Xuanhui Wang, Markus Weimer, Zhaohui Zheng, Martin Zinkevich}
%\end{compactitem}

%\subsection*{\textsc{\underline{Selected Graduate Courses}}}

%\begin{compactitem}

%\item{\textbf{Artificial Intelligence / Machine Learning}}
%%\negitemspace

%%{\small Machine Learning (R. Greiner), Reinforcement Learning
%%(R.S. Sutton), Learning and Sequential Decision Making (M.L.
%%Littman), Introduction to Control System Theory (E.D. Sontag),
%%Pattern Recognition (C. Kulikowski), Foundation of Knowledge
%%Representation (L.T. McCarty), Natural Language Processing (G.
%%Kondrak), Bounded Rationality and Non-conventional Computing (V.
%%Bulitko), Agent Communication and Architecture (R. Elio)}

%Machine Learning (Russell Greiner), Reinforcement Learning
%(Richard S. Sutton), Learning and Sequential Decision Making
%(Michael L. Littman), Introduction to Control System Theory
%(Eduardo D. Sontag), Pattern Recognition (Casimir Kulikowski),
%Foundation of Knowledge Representation (L. Thorne McCarty),
%Natural Language Processing (Greg Kondrak), Bounded Rationality
%and Non-conventional Computing (Vadim Bulitko), Agent
%Communication and Architecture (Renee Elio)

%\item{\textbf{Optimization}} %\negitemspace

%Nonlinear Programming (David F. Shanno), Discrete Optimization
%(Endre Boros), Network and Combinatorial Optimization (Shiyu Zhou)

%\item{\textbf{Algorithm / Theory}} %\negitemspace

%Design and Analysis of Algorithms (Michael L. Fredman),
%Foundations of Computer Science (Joe Kilian), Complexity of
%Computation (Andrej Bogdanov), Hard Problems and Phase Transitions
%(Joe Culberson), Automated Reasoning (Dafa Li)

%\end{compactitem}

%\subsection*{\textsc{\underline{Professional Skills}}}

%\begin{itemize}\addtolength{\itemsep}{-0.05in}

%\item{\textbf{Programming}: Primarily C/C++ and PERL, some
%experience in other popular languages}

%\item{\textbf{Systems \& Software}: Hadoop (MapReduce), MS Windows/Office, Linux/Unix,
%Mac, Matlab, \LaTeX}

%\item{\textbf{AI/ML Expertise}:}\vspace{-2mm}

%\begin{itemize} %\addtolength{\itemsep}{-0.05in}

%\item{Familiar with many popular AI/ML theories and techniques,
%such as reinforcement learning, Markov (decision) processes, decision-theoretic
%planning, statistical learning theory, PAC theory, SVMs, neural
%networks, decision trees, logistic regression, nearest neighbor learning, ensemble
%learning, Bayesian inference, EM, and online learning}

%\item{General knowledge in data mining, graphical models, feature
%selection, dimensionality reduction, unsupervised learning,
%semi-supervised learning, game theory, information theory,
%information retrieval, spam detection, anomaly detection,
%and collaborative filtering}

%\end{itemize}

%\end{itemize}

%\subsection*{\textsc{\underline{References}}}
%
%Available upon request.

%\begin{center}
%\begin{tabular}{p{35mm}p{100mm}}
%%Prof. Russell Greiner & greiner@cs.ualberta.ca \\
%%                & Department of Computing Science, University of Alberta \tabrowsep \\
%%Dr. John Langford   & jl@yahoo-inc.com \\
%%                &Yahoo! Research \tabrowsep \\
%Prof. Michael Kearns & mkearns@cis.upenn.edu \\
%               & Department of Computer and Information Science \\
%               & University of Pennsylvania  \tabrowsep \\
%Prof. Michael Littman & mlittman@cs.rutgers.edu \\
%                & Department of Computer Science \\
%                & Rutgers University \tabrowsep \\
%Prof. Ronald Parr & parr@cs.duke.edu \\
%                & Department of Computer Science \\
%                & Duke University \tabrowsep \\
%Prof. Robert Schapire & schapire@cs.princeton.edu \\
%                & Department of Computer Science \\
%                & Princeton University \tabrowsep \\
%Prof. Satinder Singh & baveja@umich.edu \\
%               & Department of Electrical Engineering and Computer Science \\
%               & University of Michigan \tabrowsep \\
%%Prof. Jason Williams & jdw@research.att.com \\
%%                & AT\&T Labs Research \tabrowsep \\
%Prof. Tong Zhang & tongz@rci.rutgers.edu \\
%                & Department of Statistics \\
%                & Rutgers University %
%\end{tabular}
%\end{center}

\end{document}
